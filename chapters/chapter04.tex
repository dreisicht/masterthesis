\chapter{Fazit}

Ziel dieser Arbeit war die Funktionsweise und die für den Einsatz der Drohne benötigten Schritte darzustellen. Der gesamte Umfang der Informationen wurde im Film innerhalb von zwei Minuten einfach und verständlich dargestellt. Hierbei konnten die Szenen mit digital erstelltem Bildmaterial, Vorgänge und Zusammenhänge besser darstellen, als mit einem Realfilm. Vorteile eines animierten Filmes sind es anhand von Illustrationen tiefere Einblicke in die Funktionsweise von Objekten zu geben. Beispielsweise wurde der Sichtkegel der Drohne im Film visuell verdeutlicht. Zudem konnte der Charakter des Flugpfades verdeutlicht werden.\\
Computeranimationen bergen zudem auch einige Nachteile, wie z.B. die Integration der digitalen Szenen in den restlichen Realfilm. Nicht zu unterschätzen ist auch der zeitliche Aufwand, der für die Erstellung einer animierten Szene notwendig ist. So wurden die vergleichsweise kurzen Szenen mit einem Zeitaufwand von etwa drei Monaten erstellt und sind damit zeitaufwändiger als das Abfilmen von realen Objekten.\\
Der im Zuge dieser Masterarbeit entstandene Film ergänzt das Portfolio des SearchWing Teams der Hochschule Augsburg im Bereich der Außendarstellung.
Der entstandene Film unterstützt dabei das Team bei der Akquise von neuen Kunden und begeisterten Teammitgliedern.