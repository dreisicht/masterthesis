
\chapter{Einleitung}

Die stetig wachsende Komplexität von Produkten und ihrer Funktionen lässt Animationsfilmen als Erklär- und Marketingmedium immer mehr Bedeutung zukommen. Die Motivation Teile eines Filmes mit Computeranimationen zu ergänzen, kann unterschiedliche Gründe haben. Kosten sind hierbei der Grund, der meist als erstes aufkommt, da kein Kamerateam zu einem eventuell weit gelegenem Punkt anreisen und in mühsamer Arbeit abfilmen muss. Im Zusammenhang mit Erklärfilmen ist jedoch die Realisierbarkeit ein viel wichtigerer Grund. So können in der Computeranimation abstrakte Zusammenhänge und nicht sichtbare technische Vorgänge für den Zuschauer einfach verständlich visualisiert werden.\\
Im Rahmen dieser Abschlussarbeit wurde ein solches komplexes Produkt behandelt. Hierbei handelt es sich um eine autonome Suchdrohne, welche an der Hochschule Augsburg entwickelt wird. Die Funktionsweise und die für den Einsatz nötigen Schritte zu beschreiben, war damit die anfängliche Zielsetzung.
Die Erstellung des Filmes ist in der folgenden Arbeit in drei Teile untergliedert -- Konzeption, Produktion und Postproduktion.