
\chapter{Einleitung}

Die stetig wachsende Komplexität von Produkten und ihrer Funktionen lässt Animationsfilmen als Erklär- und Marketingmedium immer mehr Bedeutung zukommen. Die Motivation, Teile eines Filmes mit Computeranimationen zu ergänzen, kann unterschiedliche Gründe haben. 
Insbesondere ist der finanzielle Aspekt ein wichtiger Faktor, sodass kostspielige Methoden, bei der ganze Kamerateams und ferne Orte involviert sind, nicht mehr die erste Wahl sind.
% Kosten sind hierbei der Grund, der meist als erstes aufkommt, da kein Kamerateam zu einem eventuell weit gelegenen Punkt anreisen und in mühsamer Arbeit abfilmen muss. 
Im Zusammenhang mit Infofilmen ist jedoch die Realisierbarkeit ein viel wichtigerer Grund. So können in der Computeranimation abstrakte Zusammenhänge und nicht sichtbare technische Vorgänge für den Zuschauer einfach verständlich visualisiert werden.\\
Im Rahmen dieser Abschlussarbeit wurde ein solches komplexes Produkt behandelt. Thematisiert wird hierbei eine autonome Suchdrohne, welche an der Hochschule Augsburg entwickelt wird. Diese Drohne soll hauptsächlich die Suche nach Flüchtlingen erleichtern, welche auf ihrem Weg über das Mittelmeer in Seenot geraten. Ohne dem Einsatz einer Drohne ist die Besatzung eines Seenotrettungsschiffes auf die Beobachtung mit einem Fernglas angewiesen. Damit ist der Suchradius auf etwa 5km eingeschränkt. Das SearchWing Team besteht aus etwa 12 Personen aus dem Raum Augsburg.\\
Die Anfängliche Zielsetzung des Filmes war, die Funktionsweise und die für den Einsatz nötigen Schritte für interessierte Personen zu beschreiben.
% Die Funktionsweise und die für den Einsatz nötigen Schritte zu beschreiben, war damit die anfängliche Zielsetzung.
Die Erstellung des Filmes ist in der folgenden Arbeit in drei Teile untergliedert -- Konzeption, Produktion und Postproduktion.