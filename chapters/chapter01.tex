\chapter{Konzeption}
\label{ch:intro}
In diesem Kapitel wird das Konzept des Filmes erklärt. Hierzu gehören alle grundsätzlichen Entscheidungen, wie der Inhalt des Dargestellten, die Darstellungsweise und die Zielgruppe.

\section{Zielgruppe}
\label{sec:konzept:zielgruppe}

% \graffito{Note: The content of this chapter is just some dummy text. It is not a real language.}

Als Zielgruppe wurden Kunden anvisiert, für die es in Frage kommt die Drohne einzusetzen. Dabei soll diese Zielgruppe nach dem Film wissen, wie die Drohne konkret eingesetzt wird, welche Schritte hierfür nötig sind und mit welchem Aufwand ein Einsatz einhergeht.\\
Sekundär sollen auch an dem Projekt Interessierte und potenzielle Teammitglieder eine Zielgruppe sein. Dies hatte zur Folge, dass wenige Informationen noch ergänzend eingefügt wurden, um einem interessierten Zuschauer alle nötigen Informationen zu bieten.\\
Weil der Zielgruppe, die die Drohne einsetzt, wichtig ist, dass die Drohne nicht nur ein Konzept ist, sondern schon funktionsfähig im Einsatz ist, wurde die Entscheidung getroffen den Film auf schon vorhandenem Filmmaterial aufzubauen. Somit wird gezeigt, dass die Drohne schon gebaut wurde und funktionsfähig eingesetzt wird. Die Funktionsweise ist jedoch schwer mit Filmmaterial darstellbar, weswegen ein Teil des Filmes mit Computeranimationen ergänzt wird. Beispiele für diese Erklärungen sind, der Flugpfad des Flugzeugs oder das Sichtfeld aussieht. Außerdem das Filmen im Flug der realen Drohne schwer durchzuführen.

\section{Struktur} % Outline
\label{sec:konzept:outline}

Es wurde in Rücksprache mit dem SearchWing-Team definiert, dass folgende sechs Punkte im Film in dieser Reihenfolge erscheinen sollen.

\begin{itemize}
\item{Programmieren/Flugplan}
\item{Zusammenbau}
\item{Start}
\item{Flug}
\item{Landung}
\item{Auswertung}
\end{itemize}

Zusätzlich zu diesen sechs inhaltlichen Punkten wurden ein Intro und ein Outro mit ähnlicher Bildsprache eingefügt. Somit bekommt der Film einen angenehmen Rahmen.\\
Wie unter \autoref{sec:konzept:zielgruppe} beschrieben wurde vorhandenes Filmmaterial benutzt. Da dieses Material bereits geschnittenes Material von einem Rundfunkbeitrag war, standen nur relativ kurze Stücke von Szenen zur Verfügung. Um innerhalb dieser kurzen Zeit trotzdem alle gewünschten Informationen zu vermitteln, wurde entschieden, nicht mit einem Sprecher zu arbeiten, sondern mit Texteinblendungen. Diese benötigen deutlich weniger Zeit, da hier nur Stichpunkte gelesen werden müssen, und keine vollständigen Sätze von einem Sprecher gesprochen werden müssen.
Später wurde entschieden am Ende des Filmes, im Kapitel Auswertung, ein Rettungsfloß einzufügen, um am Ende eine erfolgreiche Story dem Zuschauer vermitteln zu können.\\
Die Kapitel Flug und Landung -- abgesehen von Intro und Outro -- sind die Teile des Filmes, welche mit Computeranimationen entstanden sind. Hierbei wurde entschieden, dass ein möglichst realistischer Look angestrebt wird, damit sich diese Teile möglichst gut in das Filmmaterial einfügen.

\section{Storyboard} % Layout
\label{sec:konzept:animatic}

Der Flug wurde in einem Storyboard grob dargestellt (siehe \autoref{animatic}). In diesem Fall wurden mit einfachen 3D-Modellen pro Kameraeinstellung ein Bild erstellt. Hierzu wurde das Modell eines Segelbootes importiert und ein stark vereinfachtes Modell eines Flugzeuges eingefügt. Anschließend konnten zu den Arten der Information passende Kameraeinstellungen definiert werden. So wird beispielsweise beim Steigflug des Flugzeuges die Höhe eingeblendet, oder bei der Fluggeschwindigkeit das Flugzeug etwas weiter von Hinten gezeigt. \\
Ebenso wurde bei den Kameraeinstellungen darauf geachtet, dass stets eine Referenz aus dem vorherigen Schnitt zusehen ist. Folgendermaßen ist am Anfang der Flugeinstellung das Segelboot zu sehen, genauso auch am Ende des Fluges.\\
Für Intro und Outro wurden ein Flug über das Meer gewählt, um die Schwierigkeit der Suche eines vergleichsweise kleinen zu Rettenden darzustellen.
Diese Bilder wurden in den ersten Zusammenschnitt des Filmmaterials eingefügt. Damit wurde ein erster Eindruck erschaffen, ob der Film funktioniert und alle nötigen Informationen transportieren kann. Später wurde die Reihenfolge der Informationen und der dazugehörigen Einstellungen geändert, um eine schlüssigere Reihenfolge zu realisieren.

\begin{figure}[H]
\centering
\begin{longtable}{cc}
\subfloat[Start vom Boot]{\includegraphics[width=0.5\textwidth]{gfx/pre/0058.jpg}} &
\subfloat[Steigflug]{\includegraphics[width=0.5\textwidth]{gfx/pre/0059.jpg}} \\
\subfloat[Fluggeschwindigkeit]{\includegraphics[width=0.5\textwidth]{gfx/pre/0180.jpg}} &
\subfloat[maximalgeschwindigkeit]{\includegraphics[width=0.5\textwidth]{gfx/pre/0239.jpg}} \\
\subfloat[Flughöhe]{\includegraphics[width=0.5\textwidth]{gfx/pre/0321.jpg}} &
\subfloat[Sichtfeld]{\includegraphics[width=0.5\textwidth]{gfx/pre/0362.jpg}} \\
\subfloat[Ansicht von oben]{\includegraphics[width=0.5\textwidth]{gfx/pre/0442.jpg}} &
\subfloat[Darstellung des Flugpfades]{\includegraphics[width=0.5\textwidth]{gfx/pre/0483.jpg}} \\
\subfloat[Reichweite]{\includegraphics[width=0.5\textwidth]{gfx/pre/0565.jpg}} &
\subfloat[Rückkehr zum Boot]{\includegraphics[width=0.5\textwidth]{gfx/pre/0688.jpg}}\\

\end{longtable}
\caption{Einzelbilder des Storyboards}
\label{animatic}
\end{figure}


% \begin{figure}[H] \ContinuedFloat
% \raggedleft
% \begin{longtable}{ll}


% \end{longtable}
% \caption{Einzelbilder des Storyboards}
% \label{animatic}
% \end{figure}

